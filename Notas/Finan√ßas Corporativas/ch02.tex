\chapter{Maximizando o Valor}
\epigraph{"there is one and only one social responsibility of business—to use its resources and engage in activities designed to increase its profits so long as it stays within the rules of the game, which is to say, engages in open and free competition without deception fraud."}{Milton Friedman}
\section{O Objetivo de um Negócio}
Já asbemos que uma empresa precisa tomar decisões, mas o que rege esses decisões? O objetivo. Se um negócio não tem um objetivo a ser perseguido os gestores, se estamos falando de uma empresa de capital aberto, não tem uma parâmetro para qual decisão tomar.
\subsection*{Escolhendo um Objetivo}
Sabemos que no final o objetivo de um empresa é lucrar. Mas no curso prazo existem outros objetivos, como aumentar o \textit{marketshare}, tornar o negócio mais competitivo, etc.
\\~\\
Um objetivo ao ser escolhido determina quais projetos escolher, que tipo de dívida, entre outros. Ao mudar de objetivo todo (ou grande parte) do processo de decisão altera-se.
\\~\\
Dessa perspectiva parece que um negócio só pode escolher um único objetivo, mas isso não é verdade, existem objetivos complementares, outros são conflitantes. Se dois ou mais objetivos forem escolhidos então um deve ser prioritário, pois os dois tendo o mesmo peso pode tornar o processo de decisão conflitante.
\\~\\
Para evitar esse tipo de confusçao sçao instituídas algumas regras para a determinação dos objetivos de um negócio.
\\~\\
\textbf{1) Não-ambiguidades:} um objetivo deve ser claro e específico. O objetivo de crescer no longo prazo é ambíguo. Crescer o que? O que é longo prazo?
\\~\\
\textbf{2) Mensurabilidade:} se existe um objetivo precisamos ser capazes de mensurar suas variáveis para conseguirmos dizer se ele foi alcançado ou não.
\\~\\
\textbf{3) Sem externalidades negativas:} todo negócio tem um impacto na sociedades, alguns positivos, outros negativos. Na microeconomia chamamos esse impacto de \textbf{externalidades} \footnote{\href{https://economiamainstream.com.br/artigo/entenda-o-que-sao-externalidades/}{Entenda o que são externalidades}, por Gabriel Ferraz}. Se para alcançar seu objetivo um projeto da empŕesa implica em custo social esse deve ser internalizado por ela.
\subsection*{O Objetivo Clássico}
Friedman, em um artigo de 1970 para o New York Times
\footnote{\href{https://www.nytimes.com/1970/09/13/archives/a-friedman-doctrine-the-social-responsibility-of-business-is-to.html}{A Friedman doctrine‐- The Social Responsibility Of Business Is to Increase Its Profits}}
defende a ideia de que um negócio não tem responsabilidade social. Nessas palavras parece cotnroverso, mas a ideia central do artigo é que um negócio não tem responsabilidades em sí porque apenas pessoas possuem responsabilidades.
\\~\\
Uma corporação é um agente fictício, corporações não agem, os indivíduos que a compôem, sim. Dentro da economia isso é chamado de \textbf{Individualismo Metodológico}\footnote{\href{https://medium.com/@gabrielfferraz/individualismo-metodol\%C3\%B3gico-753f441a6d2c}{Individualismo Metodológico - Tradução por Gabriel Ferraz}}.
\\~\\
Com isso considerado considere que a responsabilidade social recaia sobre o executivo da companhia. Eke enquanto indivíduo pode apoiar todas as causas da responsabilidade social e inclusive apoiá-las com seu capital, mas enquanto exerce seu papel de executivo ele lida com dinheiro dos acionistas.
\\~\\
Assim, na visão de Friedman, única responsabilidade de um negócio é lucrar, em outras palavras, deixar seus donos mais ricos. Uma das maneiras de fazer isso é tornar as ações da empresa mais valiosas.
\\~\\
Os motivos para olhar para o preço das ações são três: 1) instantâneamente observáveis, diferente de outros indicadores os gestores recebem um feedback instantâneo de suas decisões ao observar o preço das ações; 2) os mercados são eficientes, a Hipótese dos Mercados Eficientes diz que como os agentes são racionais \footnote{A racionalidade dentro da economia tem um significado completamente diferente do senso comum. Para compreender melhor leia:
\\~\\
\href{https://economiamainstream.com.br/artigo/evolucao-racionalidade-e-transitividade/}{Evolução, racionalidade e transitividade}, por Gabriel Ferraz\\
\href{https://economiamainstream.com.br/artigo/desmistificando-falacias-sobre-homo-economicus-racionalidade-e-egoismo/}{Desmisticiando falácias: sobre homo economicus, racionalidade e egoísmo}, por Lucas Favaro} o preço de uma ação reflete as informações disponíveis de forma eficiente, ou seja, o preço da ação está sempre certo; 3) as respostas imediatas do mercado de ações são ótimas ferramentas para testar empiricamente quais projetos e como financia-los.
\subsection*{Maximizando o Preço das Ações: Cenário Ótimo}
Uma companhia, representada pelos gestores, ralaciona-se com agentes externo a si, são eles, credores, acionistas, a sociedade e o mercado financeiros. Existe um cenário ideal onde cada uma dessas relação está bem estabelecida.
\\~\\
\textbf{1) Relação diretor x acionistas:} no cenário ideal os gestores agem apenas de acordo com os interesses dos acionistas. Não por bondade nem nada disso, mas sim porque os acionistas possuem poder sobre os gestores, podendo contratá-los e demití-los quando quiserem.
\\~\\
\textbf{2) Relação diretor x credor:} os credores fornecem capital para os gestores esperando receber o montante mais os juros. Na relação ideal entre ele isso de fato acontece, o gestor pega o montante emprestado e retorna-o somado aos juros para o credor dentro do período acordado em contrato.
\\~\\
\textbf{3) Relação diretor x mercado financeiro:} o gestor garante transparência, confiança e credibilidade para o mercado financeiro em troca o mercado, pela hipótese dos mercados eficientes, reflete suas expectativas no preço das ações.
\\~\\
\textbf{4) Relação diretor x sociedade:} não existem benefícios ou custos sociais, todo custo gerado pela atividade da empresa é absorvido por ela e todo benefício social está incorporado no preço.
\subsection*{Maximizando o Preço das Ação: Realidade}
O cenário anterior é bem irreal, mas não é inútil. O valor de cenários ideais é servir de parâmetro comparativo para o que temos de fato. Como olharia para as relações como são de fato e diria "mas não deveria ser assim" se você não tivesse um cenário ótimo em mente? 
\\~\\
\textbf{1) Relação gestor x acionistas:} na realidade o gestor tem seus interesses e os coloca a frente dos dos acionistas, é comum ver empresas que mal dão lucro com altas remunerações para os executivos. Fora isso, os acionistas não possuem plenos poderes sobre os gestores.
\\~\\
\textbf{2) Relação gestor x credores:} apesar de que há uma promessa de pagamento do montante emprestado somado ao juros acordado não é bem assim que ocorre, O mercado de crédito é cheio de incertezas e isso sempre leva um aumento das taxas de juros. Quão mais arriscado é emprestar para uma empresa maior o prêmio de risco que exigo para o empréstimo. Para os credores há o risco do calote, conhecido também como \textbf{default}.
\\~\\
\textbf{3) Relação gestor x mercado financeiro:} apesar de haver a promessa de transparência e confiança não é bem assim que as coisas functionam. Não é de hoje que empresas mascaram dívidas, manipulam o lucro, etc. Mas supondo que arelação é transparente ainda há problema, existe um delay entre o fato ocorrido e chegada da informação pro mercado. Isso e outros fatores fazem do mercado acionário bem mais volátil
\\~\\
\textbf{4) Relação gestor x sociedade:} a ação da empresa na sociedade não implica em custos apenas para ela, mas para terceiros. Uma companhia de fretes inflinge nosso bem estar ao emitir carbono que pului o ar que respiramos. Isso é um exemplo clássico.
\\~\\
Caso deseje compreender melhor o problema nas relaçoes das partes com os gestores recomendo que estude sobre o modelo Principal-Agente.\footnote{Caso não tenha um \textit{background} matemático recomendo o Microeconomia, por Pindyck e Rubinfild, se tem boas noções de cálculo veja o capítulo sobre o assunto em  \textit{Microeconomic Theory}, por Nicholson e Snyder.}
\section{Governança Corporativa}
De todos os problemas descritos anteriormente o mais relevante para o nosso estudo é como os gestores relacionam-se com os acionistas.
\\~\\
Para as Finanças Corporativas a \textbf{governança corporativa} é de suma importância, no cenário ideial, como já discutimos, os gestores representam os interesses dos acionistas, mas sabemos que não é bem assim. Entender a estrutura do governança de uma empresa permite que entendamos como os interesses dos gestores são postos acima dos dos sócios.
\subsection*{Reunião Anual}
Toda empresa de capital aberto possui uma reunião anual onde os socios votam no conselho administrativo. Infelizmente, a maioria dos sócios minoritários abstém-se de votar por achar que a participação deles não faz diferença.
\\~\\
Investidores institucionais, que possuem um grande número de ações, quando instisfeito com a gestão \textit{"vote with their feet"}, vendem suas ações e vão embora. A outra parte dos grandes acionistas é mais ativa e pressiona a gestão para que representem seus interesses.
\subsection*{Conselho Administrativo}
O conselho administrativo tem o papel de gerir a companhia. Geralmente são eleitos e, em tese, atuam segundo o interesse dos acionistas.
\\~\\
Já discutimos o porblema que temos nessa relação diretores-acionistas. Saber que existe não é o suficiente, os fatores reponsáveis por diluir a responsabilidade dos acionistas são:
\\~\\
\textbf{1)} A maioria dos gestores não dedica-se 100\% à empresa, alguns por serem membros do conselho de outras companhias, outros por compromissos variado.
\\~\\
\textbf{2)} Os que consequem dedicar-se à empresa pecam em expertise na gestão. Alguns apoiam-se nos gerentes de área e até mesmo especialistas de fora da companhia.
\\~\\
\textbf{3)} Em boa parte das companhias os gestores são indicados pelo CEO (que comumente é o presidente) e não ousam questionar suas decisões.
\\~\\
\textbf{4)} Os diretoes não possuem um número relevante de ações da companhia, portanto, não se importam realmente se o preço das ações caem ou não.
\\~\\
\textbf{5)} Nos Estados Unidos o CEO, normalmente, também é o presidente da companhia. Isso é menos comum em outras partes do mundo.
\\~\\
A soma desses fatores é o caminho pelo qual o interesse dos acionistas é deixado de lado no processo de decisão de uma companhia. O CEO ser o presidente da empresa e ainda ter todo o poder coercitivo para garantir o apoio dos outros membros do conselho é uma grande falha na governança.
\\~\\
Apesar de todos esses problemas os EUA é o país onde os acionistas mais são capazes de pressionar os diretores. A capacidade de exercer poder é ainda menos em outras partes do mundo.
\subsection*{Quadro Societário}
O quadro societário de uma companhia é composto por todos aqueles que detém parte da empresa.
\\~\\
\textbf{a. Direito ao voto:} nos Estados Unidos todo indivíduo que possui uma ação tem direito à um voto. O que não é verdade no nosso caso, no Brasil e em outros países da América Latina, temos dois tipos de ações: 1) ordinárias, possuem direito ao voto, e as 2) preferenciais, não possuem direito ao voto, mas o ganho dele com dividendos é maior.
\\~\\
\textbf{b. Fundadores e sócios:} em companhias novas é comum que o fundador detenha a maior parte das ações. Os sócios podem ser internos, normalmente os diretores possuem um número grande de ações, enquanto os externos podem ser institucionais ou minoritários.
\\~\\
\textbf{c. Investidores passivos e ativos:} essa é uma divisão mais válida par aos investidores intitucionais. Os investidores passivos são aqueles que tendem a fazer o \textit{"vote with their feet"} quando a gestão é precária, normalmente fundos mútuos e de pensão
\\~\\
 Os investidores ativos são os hedge funds e private equity funds que monitoram os diretores da empresa. A presença desses sócios é um bom sinal inclusive para os investidores minoritários.
\\~\\
\textbf{d. Investidores com múltiplos interesses:} nem sempre os sócios querem o aumento do valor de suas ações. Apesar de ser estranho a primeira vista é algo recorrente. Um exemplo são as estatais, se é uma estadtal então o governo detém mais de 50\% das ações da empresa e aqui há um conflito.
\\~\\
Sendo o governo provavelmente quer arrecadar mais, porém sendo um sócio da empresa quer que o preço das ações suba. ENtão o governo, enquanto acionista, deve ponderar seus interesses.
\\~\\
\textbf{3. Outras empresas como sócias:} não é raro ver empresas que possuam ações de outras, seja por terem setores diretamente conectados, seja por questões estratégicas.
\\~\\
Nas palavras do professor Damodaran:
\begin{quote}
In summary, corporate governance is likely to be strongest in companies that have only
one class of shares, limited cross holdings and a large activist investor base and weakest
in companies that have shares with different voting rights, extensive cross holdings
and/or a predominantly passive investor base.
\end{quote}\footnote{Em resumo, a governança corporativa provavelmente será mais forte em empresas que têm apenas
uma classe de ações, participações cruzadas limitadas e uma grande base de investidores ativistas e a mais fraca
em empresas que possuem ações com diferentes direitos de voto, extensas participações cruzadas
e/ou uma base de investidores predominantemente passiva.
}
\section{Acionistas vs Diretores}
Em um cenário onde as intituições dentro da governança corportativa não garantem que os diretores representam os interresses dos acionistas temos as consequências disso. São elas:
\textbf{1. Impedindo uma aquisição hostil:} uma aquisição hostil é quando uma companhia tenta adquirir parte de outra sem que a transação seja autorizada pelos diretores da companhia alvo. A resistência dos diretores justificaca-se caso o emprego dos diretores esteja em risco, mesmo que essa aquisição seja vantajosa para os acionistas.
\\~\\
Uma das estretégias para impedir uma aquisição que conflite com o interesse dos diretores é o \textit{greenmail}. O \textit{greenmail} consiste na compra da participação da companhia aquisitora por um preço maior do que o das ações. Isso não seria um problema se fosse feito com o dinheiro dos diretores, porém, no \textit{greenmail} é a compra dessa participação utilizando o dinheiro dos acionistas da empresa alvo.
\\~\\
Outra estratégia é o \textit{golden parachute}. É uma estratégia firmada em contrato que caso o diretor seja demitido por conta de uma aquisição que garante um quantidade fixa em dinheiro durante um dado período de tempo.
\\~\\
A última estratégia é a das \textit{poison pills}. O objetivo é tornar a aquisição extremamente custosa. Uma \textit{poison pill} entra em jogo quando há uma aquisição hostil, quando isso ocorre os atuais acionistas da companhia podem comprar ações a um preço menos do que o mercado precifica. A ideia por trás disso é diluir o controle da companhia aquisitora sobre a comapnhia alvo.
\\~\\
É interessante lembrar que nenhuma dessa sestratégias para desencorajar os aquisitores hostis precisa da aprovação dos acionistas, o que não é verdade para o próximo cenário.
\\~\\
\textbf{2. Emendas anti-aquisições:} utilizar as diretrizes da empresa para dificultar que um aquisitor tenha controle sobre a empresa é  uma das estratégias possíveispara impedir uma aquisição hostil. Suponha que uma empresa exija que um sócio detenha 51\% ou mais das ações para controlar a companhia, esse é um caso de emenda anti-aquisição, há a necessidade de um desembolso muito maior para controlar isso. Assim, os diretores garantem seu poder de barganha durate o processo de aquisição.
\\~\\
\textbf{3. Pagando caro em aquisições:} num geral, aquisições não são vantajosas para os acionistas, mas são para os gestores. As justificativas para uma aquisição acima do valor das ações de uma companhia podem ser várias, os diretores podem dizer que a empresa em aquisição é mal administratada, portanto, tem pouco retorno, que os negócios possuem cinergia e/ou são companhia que complementam-se estratégicamente. Nesse caso os diretores podem ter razão e uma aquisição ter resultado positivo no longo prazo, mas via de regra, assim que uma companhia anuncia que está adquirindo outra suas ações tendem a cair.
\section{Acionistas vs Credores}
Em tese não existe conflito entre credores e acionistas. Na prática credores querem segurança e retorno do que foi acordado em contrato, os acionistas visam um \textit{upside} da empresa.
\\~\\
É natural que haja esse conflito, afinal, quando o dinheiro sai do caixa da companhia para pagar dividendos o risco recai sobre os credores pois a capacidade de pagamento da empresa diminui. Outra forma de os credores serem prejudicados é a companhia pegar o empréstimo com um projeto em mente e utilizar o capital para um projeto mais arriscado, dessa forma os juros do contrato não estão de acordo com o prêmio de risco desse empréstimo.
\\~\\
Portanto, o conflito reside na diferença na natureza do capital. Os credores recebem antes dos acionistas, são prioridade, mas apenas uma quantidade fixa, se o projeto iniciado utilizando o capital returno 5x mais do que o custo do empréstimo eles não recebem nem um bônus por isso, mas caso o projeto dê errado, eles lidam com parte significante dos custos. Os acionistas recebem o apenas o que sobra pra empresa, após várias e várias deduções, porém, se um projeto trás retornos grandes eles beneficiam-se, e muito.
\\~\\
Exemplos de conflito: 1) investimentos arriscados comumente são os que, potencialmente, trazem maiores retorno, os acionistas esperam esse retorno e apoiariam projetos arriscado, mas o credores são punidos com o risco do projeto e compensam isso cobrando juros maiores; 2) contrair mais dívidas, credores querem minimizar o risco de calote, uma empresa que alavanca-se com empréstimos pode trazer ganhos ao acionistas, mas torna-se mais arriscada e os empréstimos firmados em contrato já existentes arcam com o risco; 3) dividendos e \textit{buybacks}, a relação é simples, quando mais dinheiro sai do caixa da empresa, menor sua capacidade de pagamento e logo, mais arriscada.
\section{Firmas e o Mercado Financeiro}
Apesar de finanças corporativas serem universais e fundamentais para qualquer tipo de empresa e de qualquer tamanho creio que o leitor interessa-se mais em ser capaz da analisar aquelas listas na bolsa de valores. Nesse caso a firma é diretamente afetada pelo mercado financeiro e é como os diretores mensurariam seus resultados. Isso basea-se na \textbf{Hipótese dos Mercado Eficientes}. Sim, existe modelos de valuation que permitiriam mensurar o valor que a firma tem ou o valor garantido ao acionista, mas esses modelos precisam de muito hipóteses que muitos discordariam, seja pra valores maiores ou menores.
\\~\\
O preço da ação é resultado do agregado das expectativas baseadas nas informações disponíveis, ou seja, seria o preço justo a ser pago, isso é oelo o que os defensores da Hipótese dos Mercados Eficientes. Porém, o professor Damodaran aponta dois grandes problemas na teoria:
\\~\\
\textbf{1) Informação:} a teoria assume que a informação de um fato está imediatamente disponível após a ocorrência e que as informações são confiáveis. O problema é que as companhias postergam a publicação de ocorrências negativas. Outro problema é na confiabilidade das informação, números mascarados, manobras na contabilidade, etc, tudo isso faz com que sem uma análise aprofundada dos números (o que tomaria tempo e a teoria assume que o mercado reagiria instantâneamente à nova informação) eles levem os investidores a tomarem decisões equivocadas.
\\~\\
\textbf{2) O Problema do Mercado:} já analisamos o problema da teoria de uma das partes, da firma, mas o mercado em sí tem suas falhas. O mercado nem sempre precifica corretamente as novas informação disponíveis 

\section{Firmas e a Sociedade}
Em tese, todos os custos recaem sobre a firma e todos os benefícios estão precificados

Externalidades

Externalidades positivas: o benefício gerado pela firma não está corretamente precificado. Se a ação da firma causa um benefício maior para a sociedade é desejável que mais dele seja gerado. Existem mecanismos para estimular essa atividade. 

Externalidades negativas: o custo da produção/uso de um bem é maior do que o incorporado no produto.

O problema é que os efeitos das externalidades são difíceis de ser mensurados

\subsection*{Uma Abordagem Alternativa:}
Dados todos os fatores citados a teoria tradicional das finanças corporativas caem quando os diretores agem de acordo com seus interesses, credores não protegem-se, os mercados são ineficientes e a atuação da companhia gera altos custos para a sociedade.

A solução? Adotar uma abordagem diferente

\textbf{I. Modelo de Governança Corporativa Diferente:} 
\section{Conclusão}
