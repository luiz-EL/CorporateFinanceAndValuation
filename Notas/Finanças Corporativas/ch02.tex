\chapter{Maximizando o Valor}
\epigraph{"there is one and only one social responsibility of business—to use its resources and engage in activities designed to increase its profits so long as it stays within the rules of the game, which is to say, engages in open and free competition without deception fraud."}{Milton Friedman}
\section{O Objetivo de um Negócio}
Já asbemos que uma empresa precisa tomar decisões, mas o que rege esses decisões? O objetivo. Se um negócio não tem um objetivo a ser perseguido os gestores, se estamos falando de uma empresa de capital aberto, não tem uma parâmetro para qual decisão tomar.
\subsection*{Escolhendo um Objetivo}
Sabemos que no final o objetivo de um empresa é lucrar. Mas no curso prazo existem outros objetivos, como aumentar o \textit{marketshare}, tornar o negócio mais competitivo, etc.
\\~\\
Um objetivo ao ser escolhido determina quais projetos escolher, que tipo de dívida, entre outros. Ao mudar de objetivo todo (ou grande parte) do processo de decisão altera-se.
\\~\\
Dessa perspectiva parece que um negócio só pode escolher um único objetivo, mas isso não é verdade, existem objetivos complementares, outros são conflitantes. Se dois ou mais objetivos forem escolhidos então um deve ser prioritário, pois os dois tendo o mesmo peso pode tornar o processo de decisão conflitante.
\\~\\
Para evitar esse tipo de confusçao sçao instituídas algumas regras para a determinação dos objetivos de um negócio.
\\~\\
\textbf{1) Não-ambiguidades:} um objetivo deve ser claro e específico. O objetivo de crescer no longo prazo é ambíguo. Crescer o que? O que é longo prazo?
\\~\\
\textbf{2) Mensurabilidade:} se existe um objetivo precisamos ser capazes de mensurar suas variáveis para conseguirmos dizer se ele foi alcançado ou não.
\\~\\
\textbf{3) Sem externalidades negativas:} todo negócio tem um impacto na sociedades, alguns positivos, outros negativos. Na microeconomia chamamos esse impacto de \textbf{externalidades} \footnote{\href{https://economiamainstream.com.br/artigo/entenda-o-que-sao-externalidades/}{Entenda o que são externalidades}, por Gabriel Ferraz}. Se para alcançar seu objetivo um projeto da empŕesa implica em custo social esse deve ser internalizado por ela.
\subsection*{O Objetivo Clássico}
Friedman, em um artigo de 1970 para o New York Times
\footnote{\href{https://www.nytimes.com/1970/09/13/archives/a-friedman-doctrine-the-social-responsibility-of-business-is-to.html}{A Friedman doctrine‐- The Social Responsibility Of Business Is to Increase Its Profits}}
defende a ideia de que um negócio não tem responsabilidade social. Nessas palavras parece cotnroverso, mas a ideia central do artigo é que um negócio não tem responsabilidades em sí porque apenas pessoas possuem responsabilidades.
\\~\\
Uma corporação é um agente fictício, corporações não agem, os indivíduos que a compôem, sim. Dentro da economia isso é chamado de \textbf{Individualismo Metodológico}\footnote{\href{https://medium.com/@gabrielfferraz/individualismo-metodol\%C3\%B3gico-753f441a6d2c}{Individualismo Metodológico - Tradução por Gabriel Ferraz}}.
\\~\\
Com isso considerado considere que a responsabilidade social recaia sobre o executivo da companhia. Eke enquanto indivíduo pode apoiar todas as causas da responsabilidade social e inclusive apoiá-las com seu capital, mas enquanto exerce seu papel de executivo ele lida com dinheiro dos acionistas.
\\~\\
Assim, na visão de Friedman, única responsabilidade de um negócio é lucrar, em outras palavras, deixar seus donos mais ricos. Uma das maneiras de fazer isso é tornar as ações da empresa mais valiosas.
\\~\\
Os motivos para olhar para o preço das ações são três: 1) instantâneamente observáveis, diferente de outros indicadores os gestores recebem um feedback instantâneo de suas decisões ao observar o preço das ações; 2) os mercados são eficientes, a Hipótese dos Mercados Eficientes diz que como os agentes são racionais \footnote{A racionalidade dentro da economia tem um significado completamente diferente do senso comum. Para compreender melhor leia:
\\~\\
\href{https://economiamainstream.com.br/artigo/evolucao-racionalidade-e-transitividade/}{Evolução, racionalidade e transitividade}, por Gabriel Ferraz\\
\href{https://economiamainstream.com.br/artigo/desmistificando-falacias-sobre-homo-economicus-racionalidade-e-egoismo/}{Desmisticiando falácias: sobre homo economicus, racionalidade e egoísmo}, por Lucas Favaro} o preço de uma ação reflete as informações disponíveis de forma eficiente, ou seja, o preço da ação está sempre certo; 3) as respostas imediatas do mercado de ações são ótimas ferramentas para testar empiricamente quais projetos e como financia-los.
\subsection*{Maximizando o Preço das Ações: Cenário Ótimo}
Uma companhia, representada pelos gestores, ralaciona-se com agentes externo a si, são eles, credores, acionistas, a sociedade e o mercado financeiros. Existe um cenário ideal onde cada uma dessas relação está bem estabelecida.
\\~\\
\textbf{1) Relação gestor x acionistas:} no cenário ideal os gestores agem apenas de acordo com os interesses dos acionistas. Não por bondade nem nada disso, mas sim porque os acionistas possuem poder sobre os gestores, podendo contratá-los e demití-los quando quiserem.
\\~\\
\textbf{2) Relação credor x gestor:} os credores fornecem capital para os gestores esperando receber o montante mais os juros. Na relação ideal entre ele isso de fato acontece, o gestor pega o montante emprestado e retorna-o somado aos juros para o credor dentro do período acordado em contrato.
\\~\\
\textbf{3) Relação gestor x mercado financeiro:} o gestor garante transparência, confiança e credibilidade para o mercado financeiro em troca o mercado, pela hipótese dos mercados eficientes, reflete suas expectativas no preço das ações.
\\~\\
\textbf{4) Relação gestor x sociedade:} não existem benefícios ou custos sociais, todo custo gerado pela atividade da empresa é absorvido por ela e todo benefício social está incorporado no preço.
\subsection*{Maximizando o Preço das Ação: Realidade}
O cenário anterior é bem irreal, mas não é inútil. O valor de cenários ideais é servir de parâmetro comparativo para o que temos de fato. Como olharia para as relações como são de fato e diria "mas não deveria ser assim" se você não tivesse um cenário ótimo em mente? 
\\~\\
\textbf{1) Relação gestor x acionistas:} na realidade o gestor tem seus interesses e os coloca a frente dos dos acionistas, é comum ver empresas que mal dão lucro com altas remunerações para os executivos. Fora isso, os acionistas não possuem plenos poderes sobre os gestores.
\\~\\
\textbf{2) Relação gestor x credores:} apesar de que há uma promessa de pagamento do montante emprestado somado ao juros acordado não é bem assim que ocorre, O mercado de crédito é cheio de incertezas e isso sempre leva um aumento das taxas de juros. Quão mais arriscado é emprestar para uma empresa maior o prêmio de risco que exigo para o empréstimo. Para os credores há o risco do calote, conhecido também como \textbf{default}.
\\~\\
\textbf{1) Relação gestor x mercado financeiro:} apesar de haver a promessa de transparência e confiança não é bem assim que as coisas functionam. Não é de hoje que empresas mascaram dívidas, manipulam o lucro, etc. Mas supondo que arelação é transparente ainda há problema, existe um delay entre o fato ocorrido e chegada da informação pro mercado. Isso e outros fatores fazem do mercado acionário bem mais volátil
\\~\\
\textbf{1) Relação gestor x sociedade:} a ação da empresa na sociedade não implica em custos apenas para ela, mas para terceiros. Uma companhia de fretes inflinge nosso bem estar ao emitir carbono que pului o ar que respiramos. Isso é um exemplo clássico.
\\~\\
Caso deseje compreender melhor o problema nas relaçoes das partes com os gestores recomendo que estude sobre o modelo Principal-Agente.\footnote{Caso não tenha um \textit{background} matemático recomendo o Microeconomia, por Pindyck e Rubinfild, se tem boas noções de cálculo veja o capítulo sobre o assunto em  \textit{Microeconomic Theory}, por Nicholson e Snyder.}

\section{Hierarquia e Finanças Corporativas}

\section{Governança Corporativa}

\section{Conclusão}
